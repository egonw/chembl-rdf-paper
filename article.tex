\documentclass[10pt]{bmc_article}

\usepackage{a4wide}  % Formatting web addresses  
\usepackage{url}  % Formatting web addresses  
\usepackage{ifthen}  % Conditional 
\usepackage{multicol}   %Columns
\usepackage[utf8]{inputenc} %unicode support
%\usepackage[applemac]{inputenc} %applemac support if unicode package fails
%\usepackage[latin1]{inputenc} %UNIX support if unicode package fails
\urlstyle{rm}
%\usepackage[citestyle=numeric-comp,sorting=none,firstinits=true,doi=false,isbn=false,url=false]{biblatex}
\usepackage{cite}
%\setlength{\bibitemsep}{0cm}                                                                                                                
%\renewbibmacro{in:}{%                                                                                                                       
%  \ifentrytype{article}{}{%                                                                                                                 
%  \printtext{\bibstring{in}\intitlepunct}}}                                                                                                 
%\addbibresource{article.bib}                                                                                                                    

%\usepackage{amsmath}
%\usepackage{endnotes}
\usepackage{graphicx}
%\usepackage{tikz}

\newboolean{publ}

%Review style settings
\newenvironment{bmcformat}{\begin{raggedright}\baselineskip20pt\sloppy\setboolean{publ}{false}}{\end{raggedright}\baselineskip20pt\sloppy}

%Publication style settings
%\newenvironment{bmcformat}{\fussy\setboolean{publ}{true}}{\fussy}

\begin{document}
\begin{bmcformat}

%\pretitle{}
\title{The ChEMBL database as Linked Open Data}
 
\author{Egon~L.~Willighagen$^{1,*}$\email{Egon~L.~Willighagen - egon.willighagen@maastrichtuniversity.nl}\and
Andra~Waagmeester$^1$\email{Andra Waagmeester - andra.waagmeester@maastrichtuniversity.nl}\and
Ola Spjuth$^2$\email{Ola Spjuth - ola.spjuth@farmbio.uu.se}\and
Peter~Ansell$^3$\email{Peter Ansell - p\_ansell@yahoo.com}\and
Antony~J.~Williams$^4$\email{Antony Williams - WilliamsA@rsc.org}\and
Valery~Tkachenko$^4$\email{Valery Tkachenko - TkachenkoV@rsc.org}\and
Janna~Hastings$^5$\email{Janna Hastings - hastings@ebi.ac.uk}\and
John~Overington$^6$\email{John Overington - jpo@ebi.ac.uk}\and
Anna~Gaulton$^6$\email{Anna Gaulton - agaulton@ebi.ac.uk}\and
Mark~Davies$^6$\email{Mark Davies - mdavies@ebi.ac.uk}\and
Bin~Chen$^7$\email{Bin Chen - binchen@indiana.edu}\and
David~Wild$^7$\email{David Wild - djwild@indiana.edu}
}

\address{\iid(1)Department of Bioinformatics - BiGCaT, Maastricht University, P.O. Box 616, UNS50 Box 19, NL-6200 MD, Maastricht, The Netherlands  \\
\iid(2)Department of Pharmaceutical Biosciences, Uppsala University, PO Box 591, SE-751 24, Uppsala, Sweden \\
\iid(3)University of Queensland, St Lucia, Qld 4072, Australia \\
\iid(4)Royal Society of Chemistry, 904 Tamaras Circle, Wake Forest, NC 27587, U.S.A. \\
\iid(5)Chemoinformatics and Metabolism, European Bioinformatics Institute, POSTAL CODE, Hinxton, United Kingdom \\
\iid(6)EMBL-European Bioinformatics Institute, Wellcome Trust Genome Campus, Hinxton, Cambridgeshire, CB10 1SD, United Kingdom \\ 
\iid(7)School of Informatics and Computing, Indiana University, Bloomington, IN, U.S.A.
}

\maketitle

\begin{abstract}
\paragraph*{Background:}
Making data available as Linked Data using Resource Description Framework (RDF) promotes integration with other web resources.
RDF documents can natively link to related data, and others can link back using Universal Resource Identifiers (URIs).
RDF makes the data machine-readable, using extensible vocabularies for additional information, making it easier
to scale up inference and data analysis.
\paragraph*{Results:}
This paper describes the continued project to convert data from the ChEMBL database into RDF triples.
This updated version of ChEMBL-RDF now uses recently introduced ontologies, including CHEMINF and CiTO,
exposes more information from the database, and is now available as dereferencable, linked data.
To demonstrate these new features, we present novel use cases showing further integration with
other web resources, including Bio2RDF, Chem2Bio2RDF, and ChemSpider, and showing the use of standard
ontologies.
\paragraph*{Conclusions:}
The use cases show that by using open standards and ontologies it is possible to link the ChEMBL database
to other databases. Using those links data integration can understood by software and create a foundation
for integrative cheminformatics applications, such as the presented decision support.
\end{abstract}


%%%%%%%%%%% The article body starts:

\section*{Introduction}
\label{s1}


The current scientific data deluge, in which datasets grow faster in size and complexity than scientists can handle,
provides several new challenges for information systems. Scientists wish to 
discover new, unique, and significant patterns in datasets, which explaining biological 
phenomena not yet understood. This process requires integration of datasets that are both 
sparse, and growing independent of one another. The discovery of new patterns showing the 
causes and effects of various biological phenomena is beyond the scope of single datasets.
For example, systems biologists integrate micro-array differential expression datasets to 
biological pathways, using various other datasets to provide evidence for the links~\cite{Staab2007}. 
Another prominent example is drug discovery where a new unique chemical entity is discovered 
based on descriptions of its biological properties. This process requires the effective 
linkage of many scientific datasets~\cite{Samwald2011,Williams2012}.

ChEMBL contains descriptions for biological activities involving over a million chemical 
entities, extracted from literature. This provides a unique resource for drug researchers~\cite{Gaulton2012,Warr2009}.
It is updated on a fairly frequent basis as the existing data is further curated and new data is added. 
The ChEMBL dataset is available for download and can be browsed using a web interface. 
The former requires scientists to import the data into a relational database, while the 
latter limits the machine access to the data.

Using the right to download, modify, and redistribute the ChEMBL data, two independent teams
have previously mapped  the ChEMBL dataset to the Resource Description Framework (RDF),
Chem2Bio2RDF~\cite{Chen2010} and ChEMBL-RDF~\cite{Willighagen2011}. RDF is a framework where
knowledge is represented by small, so-called triples reflecting very basic facts. Things
in RDF are represented by Universal Resource Identifiers (URI) and when the more specialized
Universal Resource Locator (URL) is used, it creates \emph{linked data}~\cite{Samwald2011}. RDF is
developed as an open standard by the World Wide Web Consortium, and provides a knowledge
representation framework that applies to many, if not any, domain, making it a perfect
approach for data integration in the life sciences.

This paper presents an update on the ChEMBL-RDF dataset, and details of the latest structures
and ontologies used to map ChEMBL 13 to RDF, along with new use cases, showing 
links to other datasets using their RDF URIs to further support research in the life sciences.

\section*{Methods}
\label{s2}

The ChEMBL version used in this paper is ChEMBL 13, which was released on 29 February 2012.
The SQL data dump provided by ChEMBL was inserted into a local MySQL database. A set of SQL queries 
were then executed to generate RDF files using a set of PHP scripts. These scripts are Open Source 
and are available from the source code hosting service, GitHub~\cite{ChEMBLRDFGitHub}. This process 
has been outlined in a best practices note by the W3C Health Care and Life Sciences Interest
Group~\cite{Marshall2012}. This process has not changed much.

The previous ChEMBL-RDF conversion used, however, custom predicates and classes to represent RDF triples, defining
an ad-hoc ontology~\cite{Willighagen2011}. The current version, instead, uses community-proposed
ontologies, making the RDF more interoperable. The new triple dataset uses standard ontologies
including the Bibliography Ontology~\cite{Giasson2011} and the Citation Typing Ontology~\cite{Hastings2011} for literature
references and citations, and domain ontologies like the Protein Ontology\cite{Sidhu2006} and the Chemical Information
Ontology\cite{Hastings2011}. Throughout this paper, various prefixes are used to denote different ontology 
namespaces to simplify the RDF examples. These prefixes are outlined in Table~\ref{namespaces}.

\begin{table*}
\caption{Prefixes and their matching namespaces used in this paper.} \label{namespaces}
\begin{center}
\begin{tabular}{ll}
\hline
\multicolumn{2}{l}{\textbf{Common Vocabularies}} \\
bibo    & Bibliography Ontology~\cite{Giasson2011} \\
        & http://purl.org/ontology/bibo/ \\
chebi   & Chemical Entities of Biological Interest~\cite{DeMatos2010} \\
        & http://purl.org/obo/owl/CHEBI\# \\
cheminf & Chemical Information Ontology~\cite{Hastings2011} \\
        & http://semanticscience.org/resource/ \\
cito    & Citation Typing Ontology~\cite{Shotton2010} \\
        & http://purl.org/spar/cito/ \\
obo / pro & OBO \& PRotein Ontology~\cite{Sidhu2006} \\
          & http://purl.obolibrary.org/obo/ \\

\multicolumn{2}{l}{\textbf{ChEMBL-RDF Namespace}} \\
chembl & http://rdf.farmbio.uu.se/chembl/onto/\# \\

\multicolumn{2}{l}{\textbf{ChEMBL-RDF Prefixes }}\\
act    & http://linkedchemistry.info/chembl/activity/ \\
assay  & http://linkedchemistry.info/chembl/assay/ \\
mol    & http://linkedchemistry.info/chembl/molecule/ \\
res    & http://linkedchemistry.info/chembl/resource/ \\
\hline
\end{tabular}
\end{center}
\end{table*}

To expose that ChEMBL-RDF data two approaches have been adopted. First, a SPARQL end point
hosted at Uppsala University, using the Open Source Virtuoso software. Use is free, but the
querying is capped, based on the estimated computational effort. Second, resources have
been made dereferencable via the \url{http://linkedchemistry.info} resource. The Kasabi
platform~\cite{kasabi} was used for a period of time, until that project was discontinued.
The Uppsala SPARQL end point still uses resource IRIs based on this, and follow a similar
IRI pattern where linkedchemistry.info/chembl is replaced by data.kasabi.com/dataset/chembl-rdf.

Although many parts of the ChEMBL-RDF 13 dataset now use community based ontologies, some 
terms from the previous ad-hoc ontology are still used. These terms were created under the 
URI namespace \textit{http://rdf.farmbio.uu.se/chembl/onto/\#}, and are referenced here using the
prefix: \textit{chembl}.

%As a method to further test the access and possibilities of this Linked Data version of
%ChEMBL,  new uses cases have been developed for this paper.

\section*{Results}
\label{s3}

We here present the updated ChEMBL-RDF, including a description of the mapping process 
and the identification of links to other datasets.

\subsection*{Data Structure}

For each of the common resource classes, a triple pattern was defined, following the
data available in the relational database. Figure~\ref{f1} shows how the various resource
classes are linked together, detailed in this section at a triple level.

\begin{figure}[t]
\begin{center}
\includegraphics[width=0.7\textwidth]{figs/relations}
\caption{The various resource types found in the ChEMBL triples. Some entities are subclasses
of common classes, while others are instances.}\label{f1}
\end{center}
\end{figure}

The core concept in the ChEMBL database is that of the biological activity. This
is the type of information that the ChEMBL database extracts from literature.
Standard ontologies are being explored for future release for annotation
of activities, but, currently, the
type, fields and links to other resources are still using the \textit{chembl}
namespace. While ChEMBL provides both the original activities as found in literature
and standardized values allowing comparison between studies, the triples only
make the latter available. The link to literature is encoded using the CiTO ontology.
This results in a set of triples that look like:

\begin{small}
\begin{verbatim}
act:a31863
 a       chembl:Activity ;
 cito:citesAsDataSource
  res:r6424 ;
 chembl:forMolecule mol:m180094 ;
 chembl:onAssay assay:a54505 ;
 chembl:relation ">" ;
 chembl:standardUnits
  "nM" ;
 chembl:standardValue
  "100000"^^xsd:float ;
 chembl:type "IC50" .
\end{verbatim}
\end{small}

More than five thousand different activity types, represented by the chembl:type predicate,
are captured by the ChEMBL database.
The top five types are Potency (43\%), IC50 (13\%), MIC (4.6\%), Inhibition (3.7\%),
and Ki (3.6\%). The activity types in ChEMBL-RDF are currently not available as, or
using, an ontology.

The activities themselves are measured against assays, which make up a second important
resource type. Various assay types are found in the database: chembl:ADMET, chembl:Binding,
chembl:Functional, chembl:Property, and chembl:Unassigned.

\begin{small}
\begin{verbatim}
assay:a17
 a chembl:Assay ;
 cito:citesAsDataSource
  res:r11347 ;
 chembl:hasAssayType chembl:ADMET ;
 chembl:hasDescription
  "Inhibition of ... hydroxylase" .
\end{verbatim}
\end{small}

Each assay measures activities against one or more particular biological targets. To each
target the ChEMBL database associates a confidence score. Because this score is specific
for each target, the following construct is used:

\begin{small}
\begin{verbatim}
assay:a17 chembl:hasTargetScore assay:a17/score/t100122 .
assay:a17/score/t100122
 chembl:forTarget target:t100122 ;
 chembl:hasConfScore "7"^^xsd:int .  
\end{verbatim}
\end{small}

The ChEMBL database
recognizes various types: pro:PR\_0000001, chembl:ADMET, chembl:CELL-LINE,
chembl:NUCLEIC-ACID, chembl:ORGANISM, chembl:SUBCELLULAR, chembl:TISSUE,
chembl:UNCHECKED, and chembl:UNKNOWN. The latter two are currently defined as
explicit types, and effectively uses a closed-world approach. A target
specification can then look like:

\begin{small}
\begin{verbatim}
target:t1
 a chembl:Target ;
 rdfs:subClassOf pro:PR_000000001 ;
 rdfs:label "Glucoamylase" , 
  "Maltase-glucoamylase, intestinal" ;
 dc:identifier "uniprot:O43451" ,
   "3.2.1.3" ;
 dc:title "Maltase-glucoamylase" ;
 chembl:classL1 "Enzyme" ;
 chembl:hasDescription
  "Maltase-glucoamylase, intestinal" ;
 chembl:hasKeyword "Glycosidase" , 
  "Membrane" , "Sulfation" ;
 chembl:organism "Homo sapiens" .
\end{verbatim}
\end{small}

For drug discovery, the drugs themselves are the main topic of study.
ChEMBL contains many different drug types, mostly small molecules,
but also peptides, proteins, antibodies, oligosaccharides, oligonucleotides, and
even cells. ChEMBL-RDF follows the approach used by the CHEMINF ontology,
and encodes drugs as classes, rather than instances, and they subclass
classes defined by several ontologies for these drug types. For example, proteins subclass the
Protein concept from the PRotein Ontology (PR\_000000001), small
molecules subclass the Chemical Entity concept from the CHEMINF
ontology (CHEMINF\_000000), and oligosaccharides and oligonucleotides
subclass their respective matches in the CHEBI ontology (CHEBI\_50699
and CHEBI\_7754).

It should also be noted that the drugs are not subclassed from
a general drug class, or any other common superclass. However, to annotate
the drugs to be drugs, they are identified as having the 
the role of being a drug. This is triplified using the OBO and ChEBI
ontologies, where CHEBI\_23888 is the drug role:

\begin{small}
\begin{verbatim}
mol:m4 obo:has_role chebi:CHEBI_23888 .
\end{verbatim}
\end{small}

The name and synonyms for each drug are
provided as labels. When InChI and InChIKeys are available for a drug, then these are
provided via the CHEMINF formalism (here abbreviated):

\begin{small}
\begin{verbatim}
mol:m41
 rdfs:label "ChEMBL406142" , 
   "Bis(3-[1 .... yl]-propionamide)" ;
 rdfs:subClassOf cheminf:CHEMINF_000000 ;
 cheminf:CHEMINF_000200
  m41:inchikey , m41:smiles , m41:inchi ;

m41:inchikey
 a cheminf:CHEMINF_000059 ;
 cheminf:SIO_000300
   "LMCOMIDLRGMFCZ-RIPOXUOASA-N" .

m41:smiles
 a cheminf:CHEMINF_000018 ;
 cheminf:SIO_000300
  "CC(C)C[C@@H]1N2C= .... O)C7=O)NC1=O)C2=O" .

m41:inchi
 a cheminf:CHEMINF_000113 ;
 cheminf:SIO_000300
  "InChI=1S/C82H102N .... 66+,67+,68+/m1/s1" .

chemblid:ChEMBL406142
 owl:equivalentClass mol:m41 .
\end{verbatim}
\end{small}

Similarly, properties of small molecules are available from ChEMBL, and as of ChEMBL-RDF 13
these too are exposed in triple format. Like the InChI and InChIKey, these are provided using the
CHEMINF ontology approach. Here is, for example, the ALogP value:

\begin{small}
\begin{verbatim}
mol:m1
 cheminf:CHEMINF_000200 m1:alogp .

m1:alogp
 a cheminf:CHEMINF_000305 ;
 cheminf:SIO_000300
  "3.344"^^xsd:double .
\end{verbatim}
\end{small}

The literature from which all this data was extracted, is replicated from the database,
but only a subset of information about that literature is provided. In all cases,
it uses the PubMed Identifiers (PMIDs) and Digital Object Identifiers (DOIs) to link
to authoritative databases to provide full detail, as outlined later in this paper.
The provided properties are using the BIBO ontology:

\begin{small}
\begin{verbatim}
res:r1
 a bibo:Article ;
 dc:date "2004" ;
 dc:isPartOf
  journal:j6c706049c2e08871b7c46a6528065736 ;
 bibo:issue "1" ;
 bibo:pageEnd "9" ;
 bibo:pageStart "1" ;
 bibo:pmid "14695813" ;
 bibo:volume "47" .

journal:j6c706049c2e08871b7c46a6528065736
 a bibo:Journal ;
 dc:title "J. Med. Chem." .
\end{verbatim}
\end{small}

\subsection*{Data Statistics and Validation}

Each release of the ChEMBL database is accompanied by set of release counts, which help 
provide a concise content overview of the release in question. The ChEMBL 13 release counts
are as follows:
\begin{itemize}
  \item 8,845 targets
  \item 1,143,682 distinct molecules
  \item 6,933,068 activities
  \item 617,681 assays
  \item 44,682 documents
\end{itemize}

These counts are important as firstly they show continued growth in ChEMBL data, as seen 
in all previous releases. Secondly, these counts can help validate any transformations, 
which have been applied to the relational data model traditionally used to store the ChEMBL 
data. The following SPARQL queries attempt to provide a high level validation of the 
ChEMBL-RDF by regenerating the ChEMBL 13 release statistics.

The following SPARQL query returns 8,845 ChEMBL targets:
     
\begin{small}
\begin{verbatim}
select count(*) WHERE {
  ?s ?p chembl:Target .
}
\end{verbatim}
\end{small}

The following SPARQL query returns 1,143,682 ChEMBL molecules:

%Mark Davies notes:
%NOTE: Full set of ChEMBL molecules have been created by querying molecule\_dictionary table
%NOTE: I have added extra triple for each molecule: <http://linkedchemistry.info/chembl/13/molecule/m1> <http://www.w3.org/1999/02/22-rdf-syntax-ns#type> <http://rdf.farmbio.uu.se/chembl/onto/#Molecule> .
%NOTE: Modified molecule compound structure can be queried here: https://wwwdev.ebi.ac.uk/chembl/sparql

\begin{small}
\begin{verbatim}
select count(*) WHERE {
  ?s ?p chembl:Molecule .
}
\end{verbatim}
\end{small}

The following SPARQL query returns 617,681 ChEMBL assays:

\begin{small}
\begin{verbatim}
select count(*) WHERE {
  ?s ?p chembl:Assay .
}
\end{verbatim}
\end{small}

The following SPARQL query returns the 6,933,068 ChEMBL activities:

\begin{small}
\begin{verbatim}
select count(*) WHERE {
  ?s ?p chembl:Activity .
}
\end{verbatim}
\end{small}

The following SPARQL query returns the 44,681 ChEMBL documents. This is 1 less than appears 
in ChEMBL 13 release notes, as it does not include doc\_id -1, which refers to the 
unpublished dataset.

\begin{small}
\begin{verbatim}
select count(*) WHERE {
  ?s ?p bibo:Article .
}
\end{verbatim}
\end{small}

Recreation of the ChEMBL release statistics, by querying the ChEMBL-RDF with SPARQL 
provides end users with high degree of confidence that the ChEMBL-RDF contains same 
core content as source relational database. 


\subsection*{Linked Open Data}

To integrate ChEMBL-RDF with other RDF versions of scientific datasets, we link out to various resources.
These links are shown in Figure~\ref{2}. Triples for compounds link out to ChemSpider 
using the complementary index and OpenMolecules RDF using InChI values. Protein links are given to Bio2RDF~\cite{Belleau2008}
using the UniProt identifier~\cite{TheUniProtConsortium2010}. Literature references are directed to CrossRef and Bio2RDF 
using the DOIs and PubMed identifiers, respectively.

\begin{figure}[t]
{\center
\includegraphics[width=0.45\textwidth]{figs/lodgraph}
\caption{The links out of the ChEMBL-RDF data into the Linked Open Data cloud.
Edges are labeled by the predicates making the links.}\label{2}
}
\end{figure}

\subsubsection*{Linking out to Bio2RDF}

The Bio2RDF project provides both resolvable Linked Data URIs using a generic Linked Data
server~\cite{Ansell2011} to access SPARQL endpoints for a range of scientific databases~\cite{Belleau2008}.
A number of these databases are referenced in ChEMBL, including ChEBI, PubMed, and both the
UniProt protein and taxonomy databases~\cite{TheUniProtConsortium2010}. These links are
vital to provide context for use cases that require a correlation between chemical
structures and other scientific data. 

%FIXME:
%For protein links out are based on the FOOBAR for the species, the EC code for proteins,
%as well as UniProt identifiers.
%
%FIXME: should we link out to the new UniProt RDF? PA: Code is available on the OpenPHACTS branch to link to http://purl.uniprot.org/uniprot/
%
% Egon: I suggest to skip that for now, as I don't want to taint these results with Open PHACTS work

\begin{small}
\begin{verbatim}
target:t101191
 chembl:hasTaxonomy <http://bio2rdf.org/taxonomy:9606> ;
 owl:sameAs <http://bio2rdf.org/ec:2.7.11.1> ;
 owl:sameAs <http://bio2rdf.org/uniprot:Q8IYT8> .
\end{verbatim}
\end{small}

For papers with PubMed identifiers we also link out to Bio2RDF:

\begin{small}
\begin{verbatim}
res:r23 skos:exactMatch <http://bio2rdf.org/pubmed:15149661> .
\end{verbatim}
\end{small}

\subsubsection*{Linking out to ChemSpider}

ChemSpider~\cite{Pence2010} is a freely accessible, online database provided by Royal Society of Chemistry (RSC).
It contains over 26 million unique
chemical compounds aggregated from over 400 data sources as well as chemical data extracted
from RSC scientific articles and databases. Since its inception, efforts have been made to
utilize it as both a deposition platform for the community to contribute novel data, as
well as a platform for annotation and curation for existing data. Studies have shown that there are data
quality issues in many of the public compound databases~\cite{Williams2011}. ChemSpider has become a
valuable resource for curated data, especially chemical-compound name mappings. ChemSpider
is presently providing the chemical structure, substructure and similarity searching services 
underpinning the Open PHACTS semantic web project~\cite{Williams2012}. Specific chemical data 
sources containing data mappings between ChemSpider identifiers (CSIDs) and the original data 
source identifiers have been provided to the triple store, together with chemical identifiers 
including validated chemical names (systematic, generic and trivial), SMILES, and InChIs. 
The data mappings between the CSIDs and ChEMBL IDs are released to the 
community under the Creative Common Attribution-Share Alike license (CC-BY-SA 3.0). Attribution 
should be made to the original ChEMBL database, Open PHACTS, and ChemSpider.

Link mappings are provided with skos:exactMatch predicates, while the ChemSpider identifiers
are also available via a CHEMINF representation:
 
\begin{small}
\begin{verbatim}
<http://linkedchemistry.info/chembl/chemblid/ChEMBL324846>
 skos:exactMatch <http://rdf.chemspider.com/370> .
\end{verbatim}
\end{small}

\subsubsection*{Linking out to OpenMolecules RDF}

The InChI is a unique identifier for (small) organic molecules, and has been previously used
to define unique IRIs for molecules~\cite{Bradley2009,Willighagen2011}. While IRIs are theoretically unlimited in length,
in practice web browsers and servers limit the length of IRIs. Virtuoso is, unfortunately,
a system which supports only IRIs of up to a certain length. Therefore, InChI-based links are only created for smaller molecules. 
Almost 1.3 million links to OpenMolecules RDF were created in a similar manner to:

\begin{small}
\begin{verbatim}
mol:m62687 owl:equivalentClass
 <http://rdf.openmolecules.net/?InChI=1S/CH4/h1H4> .
\end{verbatim}
\end{small}

Notice here the use of owl:equivalentClass to match the formalism in CHEMINF that defines molecules
are classes, rather than instances.

\subsubsection*{Linking out to CrossRef}

In addition to the PubMed identifiers used to link from literature references to Bio2RDF,
ChEMBL provides DOIs, which we use to link out to the RDF provided by CrossRef~\cite{Bilder2011}:

\begin{small}
\begin{verbatim}
res:r2032 owl:sameAs <http://dx.doi.org/10.1016/0960-894X(96)00111-4> .
\end{verbatim}
\end{small}

\section*{Applications}

ChEMBL-RDF can be used to explore interrelated scientific datasets, and we here present a few applications. 
The first application describes how it uses the SPARQL end point can be used to make the data available as linked data
via the Bio2RDF platform. The second application takes advantage of the bibliographic information
exposed as machine readable data, in calculating citation statistics. The third application shows
an integration of ChEMBL-RDF with ChemSpider to provide an extension for the decision support
platform in Bioclipse. The last application shows how Chem2Bio2RDF can combine the ChEMBL data
with other life sciences databases, showing the power of the Linked Data approach.

\subsection*{Bio2RDF}

This first applications shows how ChEMBL-RDF can be integrated into other linked data services.
The Linked Data server used by Bio2RDF has been reconfigured for ChEMBL to provide URL
based services for standard URI resolution, along with text and link searches~\cite{Ansell2011}. 
A Java Web Archive (WAR) file along with the configuration files and build scripts for the ChEMBL-RDF Linked Data server are available on GitHub~\cite{WebAppGitHub}. 
It proxies the standard ChEMBL-RDF URIs by translating URLs between those requested by users and the URIs that are available
in SPARQL endpoints. For example, if the ChEMBL web application is running on the users
local machine, e.g. \url{http://localhost:8080/chembl/}, then a request for the
article with identifier ``a31863'', \url{http://localhost:8080/-chembl/article/a31863},
will be resolved from the database using the full original URI,
\url{http://linkedchemistry.info/chembl/activity/a31863}. If the
user requested an RDF document using content negotiation, the original URIs will be unchanged,
however, if the user requested an HTML document, the results will contain both the
original RDF triples, represented using RDFa, with links that resolve using the users local machine.

The links services enables the ChEMBL application to derive both forward links, originating in
ChEMBL, e.g. \url{http://localhost:8080/chembl/linkstonamespace/-targetns/originalns:identifier},
and backward links, originating in other databases, such as LODD, Bio2RDF,
\footnote{\url{http://bio2rdf.org/linksns/targetns/originalns:identifier}}, and Chem2Bio2RDF.
These services are vital to efficiently navigate the Linked Data web, as it is both
impractical and inefficient to require users to crawl the entire web before they can discover 
relevant resources. These services are currently only supplied as web services from ChEMBL and
Bio2RDF, but it is hoped that similar services will be provided by other scientific Linked Data
providers in the future. Datasets that are available in SPARQL endpoints can be queried for
links efficiently using simple queries as demonstrated in the ChEMBL web application.

\subsection*{Chem2Bio2RDF}

A second resource in the linked open data network in which ChEMBL-RDF has been integrated is Chem2Bio2RDF.
Chem2Bio2RDF created a single RDF repository covering over 20 public data resources pertaining to drugs, chemical compounds, carcinogens,
protein targets, genes, diseases, side effects, pathways and their relations~\cite{Chen2010}. The entities and their relations were
further annotated by Chem2Bio2OWL ontology, making it a rich semantic resource for integrative search~\cite{Chen2012a} and
mining~\cite{Chen2012b}.  ChEMBL-RDF set (version 13) has been uploaded into Chem2Bio2RDF triple store, enabling to perform
queries linking ChEMBL with other entities in Chem2Bio2RDF. Since both ChEMBL RDF and Chem2Bio2RDF use InChI keys to present chemical
compounds and adopt Bio2RDF protein identifiers to present targets, queries can be easily constructed to link each other. For instance,
to investigate the relations between drug side effects and their targets (usually off-targets), a query was created to link the
targets in ChEMBL to side effects (e.g., heart disease) in Chem2Bio2RDF via side effect related drugs and their bioassay activities.
In this case, 36 drugs causing heart diseases were linked to 87 unique protein targets (IC$_50$ $<$ 10um). The top two most common targets
of these drugs are HERG and Serotonin transporter. HERG is a well-known off-target accounting for cardiovascular problems, while
the activity of Serotonin transporter may also be associated with significant cardiovascular adverse effects~\cite{Levy2006}. 
The targets can be also linked to side effects via protein protein interactions by searching across ChEMBL, HPRD and Sider in
Chem2Bio2RDF. SPARQLs are available at \url{http://chem2bio2rdf.wikispaces.com/Integrate+with+ChEMBL} and can be executed in the
SPARQL endpoint~\cite{Chen2010}.


\begin{figure*}[!ht]
		\begin{center}
		\includegraphics[width=15cm]{cited.png}
		\newline
		\caption[wee]{Screenshot from Bioclipse Decision Support with results from a ChemSpider + ChEMBL-RDF search. The top left canvas contains the query structure, in this case the drug Carbamezapine, the top right canvas shows the near neighbors in ChemSpider (via a similarStructure search) that contain ChEMBL-RDF data, the lower right shows the chemical structure for the selected compound in the top right canvas, and the lower left canvas shows the found interactions for this compound using ChEMBL-RDF.}
	\label{fig:bioclipse-ds}
		\end{center}
\end{figure*}

\subsection*{CitedIn}

Using the SPARQL end point, it is easy to integrate data from ChEMBL into other web resources, such as in CitedIn, (http://www.citedin.org).
The impact of a scientific publication is traditionally measured by frequency of citation in subsequent publications, this narrowly defined calculation increasingly fails to capture the true reach of a paper. CitedIn was developed to make scientific impact more visible and more accurate by providing insight into where published work is being cited online in a variety of public resources. ChEMBL contains many literature references. The SPARQL endpoint is queried by CitedIn and the results are presented to its users. Figure shows the result for the Nature paper by Gamo presenting the malari structures, which is highly cited in ChEMBL~\cite{Gamo2010}. Counting citations in ChEMBL is done by the following SPARQL query.

\begin{small}
\begin{verbatim}
SELECT count(?object) WHERE {
  ?paper bibo:pmid \"$pmid\" .
  ?object cito:citesAsDataSource ?paper .
}
\end{verbatim}
\end{small}

\subsection*{Bioclipse Decision Support}

% Egon,Antony/Valery,Ola: develop and write up the Bioclipse Decision Support use case
The use of the RDF open standard also simplifies the integration of data resources with 
scientific software, as we have shown previously~\cite{Willighagen2011,Willighagen:2011kx}.
We here demonstrate a new application following this idea.

Bioclipse Decision Support (Bioclipse DS)~\cite{Spjuth:2011uq} is a user-oriented tool based on the Bioclipse 
workbench for providing on-time and on-demand information on chemical structures. 
Such information can include calculated properties, data from database queries, 
and results from predictive models. Bioclipse DS has previously been demonstrated on predictive 
modeling in drug safety assessment~\cite{Spjuth:2011uq} and also been linked to invoke and 
present results from distributed toxicity predictions from the OpenTox infrastructure~\cite{Willighagen:2011kx}.

In this study we extended Bioclipse DS with remote access to ChEMBL-RDF and ChemSpider. 
This enables users browsing chemical structures in Bioclipse to look up near neighbors in ChemSpider 
via the ChemSpider Web API (SOAP), and for the found compounds to query ChEMBL-RDF for 
available interaction data. The results are presented alongside predictive models in Bioclipse
(see Fig~\ref{fig:bioclipse-ds}), and can be used for decision support when evaluating chemical
structures and consider strategies for optimization.

\begin{figure*}[!ht]
		\begin{center}
		\includegraphics[width=15cm]{bioclipse-ds.png}
		\newline
		\caption[wee]{Screenshot from Bioclipse Decision Support with results from a ChemSpider + ChEMBL-RDF search. The top left canvas contains the query structure, in this case the drug Carbamezapine, the top right canvas shows the near neighbors in ChemSpider (via a similarStructure search) that contain ChEMBL-RDF data, the lower right shows the chemical structure for the selected compound in the top right canvas, and the lower left canvas shows the found interactions for this compound using ChEMBL-RDF.}
	\label{fig:bioclipse-ds}
		\end{center}
\end{figure*}

\subsection*{Compound Selectivity}

NOTE: May want to exclude this Use Case as query has not successful run yet, keeps causing virtuoso 
to timeout or run out of memory. Currently playing with local server setup to get query to run.

Designing a molecule, which is selective to one target over another will often be considered a 
successful outcome in a drug design process. If the target in question is a protein, it is easy
to understand that as sequence identity amongst residues, which contribute to potential small 
molecule binding sites remains high so does the issue of selectivity. That is, the molecule being 
designed may also bind to the equivalent binding site in the  closely related protein 
target and may lead to undesirable consequences. The ChEMBL data model links molecules to 
targets using different activity types recorded in the literature. Using activity types, 
which act as a measure of binding affinity, such as IC$_{50}$ or K$_{i}$ and applying activity value cuts
offs it is possible to identify molecules which have higher binding affinity to certain targets 
compared to others. Taking this one step further it is possible to identify a set molecules, 
which have a high affinity to protein A (e.g. IC$_{50}$ value $<$ 50~nM) and low affinity to protein 
B (e.g. IC$_{50}$ value $>$ 200~nM). 

The following SPARQL query identifies a set of 65 molecules, which based on data curated from the 
literature and stored in the ChEMBL-RDF data model, selectively bind Human Cyclin-Dependent Kinase 2 
(UniProt: P24941) over Human Cyclin-Dependent Kinase 4 (UniProt: P11802). 

\begin{small}
\begin{verbatim}
select distinct ?molecule WHERE {
  ?target1 owl:sameAs <http://bio2rdf.org/uniprot:P24941> .
  ?assay1 chembl:hasTarget ?target1 .
  ?activity1 chembl:onAssay ?assay1 ;
    chembl:forMolecule ?molecule ;
    chembl:type "IC50" ;
    chembl:standardUnits "nM" ;
    chembl:standardValue ?value1 ;
    chembl:relation "="
    FILTER ( ?value1 < 50 ) .
  ?target2 owl:sameAs <http://bio2rdf.org/uniprot:P11802> .
  ?assay2 chembl:hasTarget ?target2 .
  ?activity2 chembl:onAssay ?assay2 ;
    chembl:forMolecule ?molecule ;
    chembl:type "IC50" ;
    chembl:standardUnits "nM" ;
    chembl:standardValue ?value2 ;
    chembl:relation "="
    FILTER ( ?value2 > 200 ) .
}
\end{verbatim}
\end{small}

Analysis on the set of molecules returned by this query can be used to help identify small 
molecule features, which may increase target-binding specificity. For queries which link 
small molecules to targets, by traversing bioactivity data in the ChEMBL database, it is 
also important to consider the parameters associated with the assay to target mappings. 
These additional parameters include a relationship type, a multi flag (for poorly defined
targets) a complex flag (for protein complex targets) and a curation level. These different 
factors are summarized in the ChEMBL confidence score, which ranges from 9 (direct single 
protein target) to 0 (uncurated). In order to return the largest possible dataset, the 
confidence score has been ignored in this example compound selectivity use case.

\section*{Discussion}

This papers shows the evolution of the ChEMBL-RDF resource in standardizing access to the
ChEMBL data in a machine-readable manner, by using semantic web technologies. This standardization
applies to several aspects of data exchange, including the data serialization format and
access options, the data structure, and the ontologies used to explain what the data is about.
We touched upon choices made to implement these approaches, such as a selection of
RDF technologies, and chosen ontologies.

For example, for linking to other databases we selected a few predicates to link to
other databases: rdfs:seeAlso, skos:exactMatch, owl:equivalentClass, owl:sameAs. Each
has an increasing level of implications, ranging from "you may also be interested in" to
"both objects are identical and all properties apply to both objects". Additionally,
the CiTO ontology has been used to link data to the literature it was extracted from,
which provides link that lies in between rdfs:seeAlso and skos:exactMatch.

Another ontology that we adopted in this ChEMBL-RDF 13 is the cheminformatics ontology
CHEMINF, allowing us to encode numeric and non-numeric descriptors of chemical entities
in the same standardized way. Using this approach we expose identity descriptors such
as the SMILES string, InChI, and InChIKey, and molecular properties including the
logP, rotatable bond count, and hydrogen bond donor and acceptor counts.

However, not all information is yet using standardized ontologies. For example, the
activity types (IC50, EC50, Potency) are not exposed using an ontology, making it
still impossible to reason or just group over similar activity types. Likewise,
measurement units are provided as literals, making it needlessly hard to compare
concentrations in nM and mM. For this kind of data, two new ontologies were
recently developed. For example, the BioAssay Ontology addresses many of our
remaining concerns around making the experimental data machine readable~\cite{Visser2011},
while the Quantity, Unit, Dimension and Type (QUDT) ontology makes it possible
to express measurements in a machine readable way~\cite{QUDT}. Both approaches are
currently explored to be applied to ChEMBL-RDF as part of work in the
Open PHACTS project~\cite{Williams2012}.

Finally, we stress that nothing in this paper cannot be done with other approaches, such as
more common relational databases. In fact, that is exactly what the ChEMBL website does.
Similarly, there is nothing principle component analysis does that you cannot do with drawings
on paper too; it is just silly. We have here presented a linked data approach to the ChEMBL data
showing standardized and simplified integration to other resources, and that is what
cheminformatics is about.

\section*{Authors contributions}
EW initiated the project, created the initial RDF version of the ChEMBL data, and encouraged the use cases.
AW extended CitedIn to support citation info in ChEMBL-RDF and supported setting up the http://linkedchemistry.info/ host.
OS, AW, and VT developed the nearest neighbor application of Bioclipse and ChemSpider.
JH supported the project with CHEMINF representations.
PA integrated ChEMBL-RDF into Bio2RDF.
JO, AG, and MD critically validated and summarized the ChEMBL data statistics.
All authors contributed to the continued development of ChEMBL-RDF, as well as to the writing of the paper,
and approved the final version.

\section*{Acknowledgements}

The authors acknowledge A. L\"ovgren at the BMC Computing Department at Uppsala University for his
support in hosting the SPARQL end point. OS acknowledges funding from the Swedish VR (2011-6129) and eSSENCE.
The data in ChEMBL is made available by funding from the Wellcome Trust [086151/Z/08/Z].

{\ifthenelse{\boolean{publ}}{\footnotesize}{\small}
 \bibliographystyle{bmc_article}  % Style BST file
  \bibliography{article}
}

\ifthenelse{\boolean{publ}}{\end{multicols}}{}


\end{bmcformat}
\end{document}
